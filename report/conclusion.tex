

In the simulated cases with exagerated 
\begin{itemize}
	\item Proposal for further studies (Probably see if photoemmision is relevant in tenous plasma (MEO CASE, magnetospheric tail lobes))
\end{itemize}

We see a difference in the potential on the probes from the non photon emission cases versus 
the photon emission cases, and this effect is predicted. We have also shown that there 
is a difference in how great this change in potential is from probe right, to probe left.
The magnitude of this change is dependent on the direction of the plasma flow. Although this effect
is not significant in our simulations, there might be a case where it is significant if
the magnetic field or the photon emission direction is changed.