
We see a difference in the potential on the probes from the non photon emission cases versus
the photon emission cases, and this effect is predicted. We have also shown that there
is a difference in how great this change in potential is from probe right, to probe left.
The magnitude of this change is dependent on the direction of the plasma flow. Although this effect
is not significant in our simulations, there might be a case where it is significant if
the magnetic field or the photon emission direction is changed.

In the LEO regime, where the simulations have been done, the photoemitted currents are weak compared with
background plasma. Due to this the effect seen on the Langmuir probes were small, further studies should be done
in other plasma regimes where the photoemmision is more pronounced, i.e. MEO, GEO or in the magnetospheric tail lobes.
It should also be mentioned that a better numerical model for the Langmuir probes should be implemented in further
studies.