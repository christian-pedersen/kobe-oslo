
% Langmuir probes, as well as other instruments, are often situated outside of space crafts.
% The precence of the spacecraft causes disturbances to the local conditions we want to measure,
% these disturbances need to be corrected for.
% Many studies has been done on the flow of plasma around spacecraft and of the wake the
% spacecraft creates \citep{miloch_wake_2010,engwall_wake_2006}.

The spacecraft in space will always be affected by its environment, resulting various
impacts depending on the orbit (location), types of material as well as the environment
condition that changes over time \citep{trove.nla.gov.au/work/21680840}. The  most common phenomena on the spacecraft is what so
called charging. The level of charging depends on the energy of particles interacting with
the spacecraft. At the lower energy, the form of interaction of charged particles with the
spacecraft only affect the surface part called surface charging. However, the higher energy
the worse affects might be occurred on the spacecraft and in this case the charged particles
can penetrate deep inside the spacecraft component resulting in so called internal charging \citep{fennell2001spacecraft}.

Charging on the spacecraft can be simulated numerically. Numerous codes have been developed
to explain the behaviour of particles around the spacecraft as well as its interaction.
Nevertheless, it still remains many questions since the numerical simulation is only an approximation
of the real condition. However, the numerical approach has given good solutions to various applications
such as spaceflight missions. The reliability of spacecraft has been proved well before launch into space.
One of these effort is simulating the environment where the satellite will be placed during its mission.
Many parameters has been taking into account with respect to its effect on the spacecraft.
The results of the simulation can be significant point for decision maker for the spaceflight
mission. This is one of reasons why this simulation is the primary interest for this study.

In this study, we attempt to simulate a spacecraft named Norsat-1 which will be placed in
Low Earth Orbit (LEO) environment around 600 km altitude and polar inclination around \(98.8^\circ\).
Initially this satellite will be launched in early 2016
\citep{norSat}. This satellite
has been equipped with two probes. It is interesting since this satellite will pass over
the auroral region more frequently. In this region  the satellite will be exposed not only to
rapid variation of the thermal component of the ionosphere \citep{hastings1995review}, but also to high
energy particles from the solar wind.

Since we use the EMSES (Electro Magnetic Spacecraft Environment Simulator) code \citep{miyake_plasma_2013} in the
simulation, it is important to point out that only the effects of background plasma as
well as the photoelectrons from the sunlight are taken into account. The simulation
has been done into several cases and each case has been grouped into two, i.e. plasma
flows with and without the photoemission effects on the spacecraft as well as the probes.
All results will be presented in detail in the specific section in this report.
