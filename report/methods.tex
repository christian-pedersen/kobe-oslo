\begin{itemize}
	\item Short PiC (EMSES) explanation
	\item Experimental set up
	\item 
\end{itemize}
\subsection{Numerical methods}

To solve the problem numerically we use the EMSES code. EMSES uses the standard PIC method for plasma simulations.
In the code we are able to define a spacecraft body, and the code then calculates the potential on that body using the capacitance matrix method.
Although EMSES has the capability to do a full electromagnetic calculation, we have opted to use the poisson's equation 
solver for electrostatic problems. In the EMSES system we can define sunlit surfaces based upon an angle, and a current 
desity. Sunlit surfaces will then emmit electrons based upon a energy distribution. For a complete description of EMSES' capabilities
see \citep{nakashima_ohhelp:_2009} Parameters are choosen to simulate the earth, but with an enhanced flux to emphasize the effect in question. 


\subsection{Test case setup}

We wish to simulate the effects of Photon emmision in different test cases, and have thus set up the following
6 cases:
\begin{center}
    \begin{tabular}{ | l | l | l | p{5cm} |}
    \hline
    Case & Plasme flow & Photon emission  \\ \hline
     1: & 41600 $\vec{e_x}$ m/s  & 0 \\ \hline
     2: & -41600 $\vec{e_z}$ m/s & 0 \\ \hline
     3: & -41600 $\vec{e_y}$ m/s & 0 \\ \hline
     4: & 41600 $\vec{e_x}$ m/s & $-10^{-3} A/m^{3}$ $\vec{e_x}$\\ \hline
     5: & -41600 $\vec{e_z}$ m/s & $-10^{-3} A/m^{3}$ $\vec{e_x}$\\ \hline
     6: & -41600 $\vec{e_y}$ m/s & $-10^{-3} A/m^{3}$  $\vec{e_x}$\\
    \hline
    \end{tabular}
\end{center}

So test case 1-4, 2-5, and 3-6 are the ``same'' cases exept that we run the simulation with and without
photon emission to compare the cases two and two. 