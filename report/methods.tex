\subsection{Numerical methods}

To solve the problem numerically we use the EMSES code. EMSES uses the standard PIC method for plasma simulations \citep{birdsall2004plasma}.
In the code we are able to define a spacecraft body, and the code then calculates the potential on that body using the capacitance matrix method.
Although EMSES has the capability to do a full electromagnetic calculation, we have opted to use the Poisson's equation
solver for electrostatic problems. In the EMSES system we can define sunlit surfaces based upon an angle, and a current
density. Sunlit surfaces will then emit electrons based upon a energy distribution. For a complete description of EMSES' capabilities
see \citep{nakashima_ohhelp:_2009}. Parameters are choosen to simulate the sun at the earth, but with an enhanced flux to emphasize the effect in question.

\subsection{Theoretical calculations}
\label{sec:theo_calc}

If there are big difference between spacecraft size and the Debye length of plasma, we can evaluate the theoretical potential of the spacecraft($\Phi_d$). In order to get $\Phi_d$, we need the ion, electron and photoelectron current (\(I_i\), \(I_e\) and \(I_{ph}\)).\\
In the case that the value of $\Phi_d$ is considered to be negative, each current is written as below:

\begin{equation}
\begin{split}
 I_i = &
    \left\{\begin{array}{ccc}
       A|q|n_\infty\sqrt{\frac{8 k T_i}{\pi m_i}} (without flow)\\
       \frac{1}{6}A |q|n_\infty V + \frac{5}{6} A |q| n_\infty \sqrt{\frac{8 k T_i}{\pi M_i}}(with flow)
      \end{array}\right. \\
  I_e = & -A|q|n_\infty \sqrt{\frac{8 k T_e}{\pi m_e}}exp\Big(\frac{|q|\Phi_d}{k T_e}\Big)\\
  I_{ph} = & \frac{1}{6} AJ_s
\end{split}
\label{thin sheet potential}
\end{equation}
where \(A\) is the surface area of the spacecraft, \(|q|\) is elementary charge, \(n_\infty\) is the plasma density, \(k\) is Boltzmann's constant, \(T_i\)/\(T_e\) is the temperature of ion/electron, \(M_i\)/\(M_e\) is the mass of ion/electron and \(J_s\) is the photoelectron density.
Submitting these equations to \(I_i+I_e+I_{ph}=0\) and solving for $\Phi_d$, we can get
\begin{equation}
\begin{split}
 \Phi_d = & \frac{k T_e}{|q|}ln\Big(\sqrt{\frac{T_i}{T_e}}\sqrt{\frac{m_e}{m_i}}+\frac{J_s}{6 n_\infty |q|}\sqrt{\frac{\pi m_e}{8 k T_e}}\Big) + \Phi_0 \\
 \Phi_d = & \frac{k T_e}{|q|}ln\Big(\frac{5}{6}\sqrt{\frac{T_i}{T_e}}\sqrt{\frac{m_e}{m_i}}+\frac{V}{6}\sqrt{\frac{\pi m_e}{8 k T_e}} + \frac{J_s}{6 n_\infty |q|}\sqrt{\frac{\pi m_e}{8 k T_e}} \Big)+\Phi_0 \\
 \Phi_0 = & k T_{ph}, \ V = speed\ of\ flow
\end{split}
\label{thin sheet potential 2}
\end{equation}
where \(T_{ph}\) is the temperature of photoelectron.
The theory is based on thin-sheath approximation, so if spacecraft size is much larger than the Debye length of plasma, we can use it.\\

\subsection{Test case setup}

We wish to simulate the effects of Photon emitted electrons in different test cases, and have thus set up the following
6 cases:
\begin{table}
\begin{center}
    \begin{tabular}{ | l | l | l | p{5cm} |}
    \hline
    Case & Plasma flow (\(V\)) & Photon emission  \\ \hline
     1: & 41600 $\vec{e_x}$ m/s  & 0 \\ \hline
     2: & -41600 $\vec{e_z}$ m/s & 0 \\ \hline
     3: & -41600 $\vec{e_y}$ m/s & 0 \\ \hline
     4: & 41600 $\vec{e_x}$ m/s & $-10^{-3} A/m^{3}$ $\vec{e_x}$\\ \hline
     5: & -41600 $\vec{e_z}$ m/s & $-10^{-3} A/m^{3}$ $\vec{e_x}$\\ \hline
     6: & -41600 $\vec{e_y}$ m/s & $-10^{-3} A/m^{3}$  $\vec{e_x}$\\
    \hline
    \end{tabular}
\end{center}
	\caption{The six different cases we investigate, \(1-3\) is without a photoemission, and \(4-6\) is the same cases with photoemission.}
\end{table}

	\begin{table}[h]
		\centering
	    \begin{tabular}{ | l | l | l| l| l | l | l |}
	    \hline
		Stepsize & Timestep & Density &\(|B|\) & \(T_e\) & \(T_i\) & N\(_x\)\\
		\hline
		\(1.0\)cm & \(5\text{E}-10\)s & \(1.0\text{E}5\)kg/cm\(^3\) &  \(50\text{E}-6\) T & \(3000\)K & \(1500\)K  & \(128\)\\
	    \hline
	    \end{tabular}
		\caption{Input parameters in EMSES}
	\end{table}

So test case 1-4, 2-5, and 3-6 are the ``same'' cases except that we run the simulation with and without
photon emission to compare the cases two and two. We define 3 geometric objects, the spacecraft itself
and two probes. In all cases the $B$ field is in the $\vec e_z$ direction.

\begin{figure}
        \includegraphics[width = \textwidth]{images/picture_simulation1.png}
	\caption{(a)Geometry of spacecraft and surroundings. Figure is no P-E simulation cases.}
        \includegraphics[width = \textwidth]{images/picture_simulation2-2.png}
        \caption{(b)Geometry of spacecraft and surroundings. Figure is P-E simulation cases.}
    \end{figure}
