In order to measure the characteristic value of each particle, we set probes in the
place that the wake is formed or not.
We set the size of the Probes to 1 $cm^3$ in this simulation. The probes are represented as a singular
grid point in the simulation, due to this and the interpolation the ion and electron densities
are not \(0\), inside the probes. And they are not a real
physical model. We need to simulate a variety of sizes. A similar treatment of the probes, considering the surrounding surface,
as the thin wire booms in \citet{miyake_plasma_2013}.
(For example, if we give probes the photoelectric effect or if we change the size of satellite and
probes to realistic scales, the results might be changed.)
We see a difference in cases, so it might be a case we have not considered that that is significant.

Some earlier papers have been written on the effects of photoemmission showing that the charging effect
on a spacecraft is significant.\citep{ergun_spacecraft_2010} But in these papers the objects are in wery different
plasma conditions than in our cases. So we conclude that in similar cases to ours the photoemission effect
will not be significant. We should note here that there may still be a worst case scenario for the Norsat-1
that is not considered in this paper, so we can not exlude this possibility, especially scince we se an effect and an
asymetry in the potentials of the probes based on the direction of the flow. But we can state that if
it is the case for two probes that they are wery sensitive to unexpected changes in potentials, symetric or asymetric form
right to left, then the effect of photoemission must be accounted for.
