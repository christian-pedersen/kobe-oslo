

We compared cases with photoemmission and without photoemmission. We think whether photo emission affect the Probes.
We set the size of the Probes to 1 $cm^3$ in this simulation. But it is the same size as the grid.
So density does not become 0 in the inner of Probes. We need to simulate a variety of sizes.
We try the case of more number of grid, longer a grid, and size of Probes.
In order to measure the characteristic value of each particle, we set probes in the
place that the wake is formed or not. But they are not a real physical model.
(For example, if we give probes the photoelectric effect or if we change the size
of satellite and probes to realistic scales, the results might be changed.)
We see a difference in cases, so it might be a case we have not considered that that is significant.

Some earlier papers have been written on the effects of photoemmission showing that the charging effect 
on a spacecraft is significant.\citep{Ergun} But in these papers the objects are in wery different
plasma conditions than in our cases. So we conclude that in similar cases to ours the photoemission effect
will not be significant. We should note here that there may still be a worst case scenario for the Norsat-1
that is not considered in this paper, so we can not exlude this possibility, especially scince we se an effect and an
asymetry in the potentials of the probes based on the direction of the flow. ----Altough, we have found the effect to be small 
there is still a valid question how small of a change in the potentials, especially in the case where we have an asymetry in
the change on the probes can we have before it is noticable for data sent back from the spacecraft.---But we can state that if 
it is the case for two probes that they are wery sensitive to unexpected changes in potentials, symetric or asymetric form 
right to left, then the effect of photoemission must be accounted for.