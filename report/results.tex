\begin{itemize}
	\item Comparison of cases with P-E and without
	\item Acc of charges at probes + $\phi$, diff flows and $\alpha$
	\item $\phi$ num vs ana
\end{itemize}

\subsection{Induced electric current}

	%Redo this part if necessary
	The plasma is flowing in in relation to the coordinate system in the simulations.
	Due to this an induced electrical field, \(\varepsilon\), will appear. To analyze the potential we
	want to correct for this potential field. The induced electrical field will
	neutralize the Lorentz force. Combined with the electrostatic approximation
	we can obtain the \(\varepsilon\)

	\begin{equation}
		\vec{\varepsilon} = \vec{v_D}\times \vec{B}
	\end{equation}

	\begin{equation}
		\int{Edx} = -\phi
	\end{equation}

	\begin{equation}
		\phi = -\int \vec{v}_d\times\vec{B} \approx -\int \left( 41600 \text{m/s}\cdot 50E-6 \text{T} \right) dx
	\end{equation}
	\begin{equation}
		\phi = 2.08 \text{x}
	\end{equation}

\subsection{Photoemmision paths}

	\begin{figure}
		\includegraphics[width = 0.49 \textwidth]{images/case6_jph_paths}
		\includegraphics[width = 0.49 \textwidth]{images/case6_jph_paths_2}
		\caption{The trajectories of the electrons emmited by the photoelectric effect in simulation \(6\). It can be seen that some
		of the trajectories coincide with the volume occupied by the langmuir probes. The electrons are strongly affected by the magnetic
		field \(\vec{B}\), and follows a gyrating path.}
 	\end{figure}
