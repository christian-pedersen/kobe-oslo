\subsection{Induced electric current}
	%Redo this part if necessary
	The plasma is flowing in in relation to the coordinate system in the simulations.
	Due to this an induced electrical field, \(\varepsilon\), will appear.
	The induced electrical field will neutralize the Lorentz force.
	Combined with the electrostatic approximation we can obtain the \(\varepsilon\)

	\begin{equation}
		\vec{\varepsilon} = \vec{v_D}\times \vec{B}
	\end{equation}

	This will cause a potential gradient perpendicular to the plasma flow and the magnetic field,
	using the electrostatic approximation we obtain the magnitude of the gradient.

	\begin{equation}
		\int{Edx} = -\phi
	\end{equation}

	\begin{equation}
		\phi = -\int \vec{v}_d\times\vec{B} \approx -\int \left( 41600 \text{m/s}\cdot 50E-6 \text{T} \right) dx
	\end{equation}
	\begin{equation}
		|\nabla\phi| = 2.08 \text{m}^{-1}
	\end{equation}

	\begin{figure}
		\includegraphics[width = \textwidth]{images/emph}
		\caption{The blue line is the potential along direction x for simulation \(6\). In this case the potential gradient is along
		the \(x-\)axis. The dotted green line is the potential caused by the induced electrical field. This should be accounted for
		if  we want to find the potential at the spacecraft and the probes.}
		\label{fig:emph}
	\end{figure}

	Figure~\ref{fig:emph} shows the measured potential at case \(6\).


\subsection{Photoemmision paths}

	The electrons emmited from the spacecraft due to the photoelectric effect, have a kinetic
	energy corresponding to a Maxwellian distribution with a temperature of \(T_{ph} =  3.8481\cdot 10^{4} \text{K}\).
	Figure~\ref{fig:trajectories} illustrates the trajectories of the emmited electrons in simulation \(6\).
	As the probes are situated \(10 \text{cm}\) to the sides of the spacecraft on the \(x-\)axis, the probes
	may be hit by the photo-emmited electrons. In the following section, \ref{sec:acc_emmited}, we show the number of electrons hitting
	the probes.


	\begin{figure}
		\includegraphics[width = 0.49 \textwidth]{images/case6_jph_paths}
		\includegraphics[width = 0.49 \textwidth]{images/case6_jph_paths_2}
		\caption{The trajectories of the electrons emmited by the photoelectric effect in simulation \(6\). The possible
		paths of the photoemmited electrons coincide with the volume occupied by the langmuir probes. The photoemmited electrons are strongly affected by the magnetic
		field \(\vec{B}\), and follows a gyrating path guided by \(\vec{B}\). The photoemmited electrons are in all the studied cases
		emmited from the spacecraft in \(-x\) direction, and the paths are similar. The langmuir probes are situated \(10 \text{cm}\) to each side
		of the spacecraft along the \(x-\)direction. (NOTE, should have axis labels, and domain length.)}
		\label{fig:trajectories}
	\end{figure}

\subsection{Potential difference with P-E and no P-E}

\subsubsection{Case 1 vs case 4}

\begin{figure}
    \includegraphics[width = 0.3 \textwidth]{images/pot_case14.png}
    \caption{Potential of satelite and surroundings in the x-direction for case 1 and case4.}
\end{figure}


Here we have the emitted electrons in the negative flow direction. As expected this
leads to a drop in potential in the left probe which is facing the plasma flow. The
potential drop over the probe is 3.8\%. The right probe is now the wake where we have
a drop in the ion density. This yields a large drop in potential compared to the left
probe, but it also has a larger than the left probe when comparing case 1 and 4. This
might be because the potential drop on the left side redirects more ions from the right side.
The potential drop comparing the two cases is 10\%. The potential rise over the satelite is 28\%.


\subsubsection{Case 2 vs case 5}


\begin{figure}
    \includegraphics[width = 0.5 \textwidth]{images/pot_case25.png}
    \caption{Figure show potential of satelite and surroundings in the x-direction for case 2 and 5.}
    \label{fig:pot_case25}
\end{figure}

With the flow of emmited electrons in the negative x direction we would expect a rise in electron density around the left probe.
This can be seen in figure \ref{fig:pot_case25} where we see a 5.4\% drop in the potential of this probe compared to case 5 with no emitted
electrons. On the right probe we have a small rise in the potential of 3.3\%. With no emitted electrons on this side of the satelite the rise in
potential can be explained by looking at the increase in ion density as seen in figure (need ref).
In the satelite we have a potential rise of 28\%. So the change in potential in the probes are small compared to the change in the satelite.

\subsubsection{Case 3 vs case 6}

\begin{figure}
    \includegraphics[width = 0.5 \textwidth]{images/pot_case36.png}
    \caption{Figure show potential of satelite and surroundings in the x-direction for case 3 and 6.}
\end{figure}

A rather large drop in potential of \( 12\)\% on both probes. Potential rise of \(26\)\% over the satelite.

\subsection{Wake plots}

A rather large drop in potential of 12\% on both probes. Potential rise of 26\% over the satelite.


    \begin{figure}
        \includegraphics[width = 0.3 \textwidth]{images/ion_density_case1}
        \includegraphics[width = 0.3 \textwidth]{images/ion_density_x-z_case2}
	\includegraphics[width = 0.3 \textwidth]{images/ion_density_x-y_case3}
        \caption{Ion density of spacecraft and surroundings without P-E Figure on the left displays case 1. Middle figure displays case 2. Rightmost figure displays case 3.}
    \end{figure}
